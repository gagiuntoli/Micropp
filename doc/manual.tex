\documentclass[conference]{IEEEtran}
% \usepackage[pdftex]{graphicx}
% \usepackage[dvips]{graphicx}
%\usepackage{amsmath}
\usepackage{algorithm}
\usepackage{algorithmic}
%\usepackage{array}
%\usepackage[caption=false,font=normalsize,labelfont=sf,textfont=sf]{subfig}
%\usepackage{fixltx2e}
%\usepackage{stfloats}
%\usepackage{url}
\usepackage{amsmath}
\usepackage{amsfonts}
\usepackage{amssymb}

\hyphenation{op-tical net-works semi-conduc-tor}


\begin{document}

\title{MicroPP: Reference Manual}

\author{\IEEEauthorblockN{
	Guido Giuntoli\IEEEauthorrefmark{1}\IEEEauthorrefmark{2},
	Jimmy Aguilar\IEEEauthorrefmark{1}\IEEEauthorrefmark{3}}
\IEEEauthorblockA{\IEEEauthorrefmark{1}Barcelona Supercomputing Center}
\IEEEauthorblockA{\IEEEauthorrefmark{2}guido.giuntoli@bsc.es}
\IEEEauthorblockA{\IEEEauthorrefmark{3}jimmy.aguilar@bsc.es}}

\maketitle

% no keywords

\IEEEpeerreviewmaketitle

\section{Introduction}
% no \IEEEPARstart

\hfill August 26, 2015

\section{Implementation}

The Voigt convention used here is the same as in Ref.~\cite{simo}.
\begin{equation}
\epsilon = \left[\epsilon_{11} \quad \epsilon_{22} \quad \epsilon_{33} \quad \epsilon_{12} \quad \epsilon_{13} \quad \epsilon_{23} \right]^T
\end{equation}

\section{Geometries}

\section{Material Models}

\subsubsection{Plasticity Model}

MicroPP has a J2 plasticity model with isotropic hardening.

\begin {equation}
\left\{
\begin{array}{ll}
\epsilon_{n+1}^{e,\text{trial}} = \epsilon_{n+1} - \epsilon_{n}^{p} \\[5pt]
\sigma_{n+1}^{\text{trial}} = \mathrm{C} : \epsilon_{n+1}^{e,\text{trial}} \\[5pt]
\mathbf{q}_{n+1}^{\text{trial}} = \mathbf{q}_{n} \\[5pt]
f_{n+1}^{\text{trial}} = f (\sigma_{n+1}^{\text{trial}}, \mathbf{q}_{n+1}^{\text{trial}})\\
\end{array}
\right.
\end {equation}

\begin {equation}
\left\{
\begin{array}{ll}
\epsilon_{n+1}^{p} = \epsilon_{n}^{p} - \Delta \gamma  \mathbf{n}_{n+1} \\[5pt]
\alpha_{n+1} = \alpha_{n} + \sqrt{\frac{2}{3}} \Delta \gamma
\end{array}
\right.
\end {equation}

\begin {equation}
s_{n+1}^{\text{trial}} = s_{n+1} + 2 \mu e_{n+1}
\end {equation}

\begin {equation}
\mathbf{n}_{n+1} = \frac{s_{n+1}^{\text{trial}}}{|| s_{n+1}^{\text{trial}} ||}
\end {equation}

\begin {equation}
f_{n+1}^{\text{trial}} = || s_{n+1}^{\text{trial}} || - \sqrt{\frac{2}{3}} (\sigma_{Y} + K_{a} \alpha_{n} )
\end {equation}

\begin {equation}
\Delta \gamma = \frac{f_{n+1}^{\text{trial}}}{2\mu(1+K_{a})}
\end {equation}

1. Compute trial elastic stress
\begin {equation}
\left\{
\begin{array}{ll}
e_{n+1} = \epsilon_{n+1} - \frac{1}{3} \text{tr}( \epsilon_{n+1} ) \\[5pt]
s_{n+1}^{\text{trial}} = 2\mu( e_{n+1} - e_{n+1}^{p} ) \\[5pt]
\end{array}
\right.
\end {equation}

2. Check yield condition

\begin {equation}
f_{n+1}^{\text{trial}} = || s_{n+1}^{\text{trial}} || - \sqrt{\frac{2}{3}} (\sigma_{Y} + K_{a} \alpha_{n} )
\end {equation}

3. If $f_{n+1}^{\text{trial}} < 0$ then set
\begin {equation}
(\cdot)_{n+1} = (\cdot)_{n+1}^{\text{trial}} \text{ and exit.}
\end {equation}

\begin {equation}
\Delta \gamma = \frac{f_{n+1}^{\text{trial}}}{2\mu(1+K_{a})}
\end {equation}

4. Compute
\begin {equation}
\mathbf{n}_{n+1} = \frac{s_{n+1}^{\text{trial}}}{|| s_{n+1}^{\text{trial}} ||}
\end {equation}

\begin {equation}
\Delta \gamma = \frac{f_{n+1}^{\text{trial}}}{2\mu(1+K_{a})}
\end {equation}

5. Update variables
\begin {equation}
\left\{
\begin{array}{ll}
\epsilon_{n+1}^{p} = \epsilon_{n}^{p} - \Delta \gamma  \mathbf{n}_{n+1} \\[5pt]
\alpha_{n+1} = \alpha_{n} + \sqrt{\frac{2}{3}} \Delta \gamma \\[5pt]
\sigma_{n+1} = k \, \text{tr} (\epsilon_{n+1}) + s_{n+1}^{\text{trial}} - 2 \mu \Delta \gamma \mathbf{n}_{n+1} 
\end{array}
\right.
\end {equation}

\begin{algorithm}
\caption{Algorithm caption}
\label{alg:algorithm-label}
\begin{algorithmic}
 \STATE $ \epsilon^{0} = \epsilon $
 \STATE $ \sigma^{0} = g(\epsilon^{0})$
 \FOR {$i=1\dots6$}
     \STATE $ \epsilon^{*} = \epsilon $
     \STATE $ \epsilon^{*}(i) = \epsilon^{*}(i) + \delta\epsilon $
     \STATE $ \sigma^{*} = g(\epsilon^{*})$
     \STATE $ \mathrm{C} (i, :) = ( \sigma^* - \sigma^0 ) / \delta\epsilon $
 \ENDFOR
\end{algorithmic}
\end{algorithm}


\section{Conclusion}
The conclusion goes here.

% use section* for acknowledgment
\section*{Acknowledgment}

The authors would like to thank...

\begin{thebibliography}{1}
\bibitem{simo} J.C. Simo \& T.J.R. Huges.\emph{Computational Ineslasticity}, Springer, 2000. 
\bibitem{oller} S. Oller. \emph{Numerical Simulation of Mechanical Behavior of Composite Materials}, Springer, 2014. 
\end{thebibliography}

\end{document}
