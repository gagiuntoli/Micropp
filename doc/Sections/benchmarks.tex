
\section{Benchmarks}

\subsection{\texttt{benchmarks-sol-ass}}

This benchmark is intended to measure the computing times of the assembly of the Residue vector ($\times2$) and the
Jacobian matrix and the solver (CGPD).

Basically the benchmark executes a tipical Newton-Raphson iteration:

\begin{lstlisting}[language=c++, backgroundcolor=\color{lightgray} ]
	double norm = assembly_rhs_acc(u, nullptr, b);

#ifdef _OPENACC
	assembly_mat_acc(&A, u, nullptr);
#else
	assembly_mat(&A, u, nullptr);
#endif

#ifdef _OPENACC
	int cg_its = ell_solve_cgpd_acc(&A, b, du, &cg_err);
#else
	int cg_its = ell_solve_cgpd(&A, b, du, &cg_err);
#endif

	for (int i = 0; i < nn * dim; ++i)
		u[i] += du[i];

	norm = assembly_rhs_acc(u, nullptr, b);
\end{lstlisting}

To execute this benchmark:

\begin{lstlisting}[language=Bash,backgroundcolor=\color{lightgray} ]
./benchmark-sol-ass
\end{lstlisting}

\begin{figure}[!htbp]
	\centering
	\begin{tikzpicture}[]
		\pgfplotsset{every tick label/.append style={font=\small}}
		\pgfplotstableread{data/benchmark-sol-ass-macintosh-intel-I7-3520M.dat}{\mac}
		\pgfplotstableread{data/benchmark-sol-ass-CTEPOWER-PGI19.4-cpu.dat}{\ibmcpu}
		\pgfplotstableread{data/benchmark-sol-ass-CTEPOWER-PGI19.4-gpu.dat}{\ibmgpu}
		\begin{loglogaxis}[
			grid=major,
			y unit=s,
			legend style={at={(1.7,0.45)},anchor=south east},
			legend cell align={left},
			ylabel=Computing Time,
			xlabel=Micro-Scale Mesh Resolution,
			x unit=\# Elements,
			scaled y ticks=false,
			%xmin=-1,
			%ymin=-1,
			xtick = {20,40,60,80,100},
			xticklabels = {20\tst,40\tst,60\tst,80\tst,100\tst},
			]
			\addplot [color=blue,mark=*,line width = 0.2mm] table [x index={0}, y
				expr=\thisrowno{1}*1.0e-3] {\mac};
			\addplot [color=green,mark=*,line width = 0.2mm] table [x index={0}, y
				expr=\thisrowno{2}*1.0e-3] {\mac};
			\addplot [color=blue,mark=+,line width = 0.2mm] table [x index={0}, y
				expr=\thisrowno{1}*1.0e-3] {\ibmcpu};
			\addplot [color=green,mark=+,line width = 0.2mm] table [x index={0}, y
				expr=\thisrowno{2}*1.0e-3] {\ibmcpu};
			\addplot [color=blue,mark=x,line width = 0.2mm] table [x index={0}, y
				expr=\thisrowno{1}*1.0e-3] {\ibmgpu};
			\addplot [color=green,mark=x,line width = 0.2mm] table [x index={0}, y
				expr=\thisrowno{2}*1.0e-3] {\ibmgpu};
			\legend{Assembly (Intel I7 CPU), Solver (Intel I7 CPU),
			Assembly (IBM P9 CPU), Solver (IBM P9 CPU),
			Assembly (V100 GPU), Solver (V100 GPU)}
		\end{loglogaxis}
	\end{tikzpicture}
	\caption{\label{fig:ass_vs_sol}
		Computing time used for the assembly of the Jacobian Matrix and the Residue vector and
		the solver algorithm of the Micropp code to perform the micro-scale FE
		calculation in an Intel Core i7-3520M CPU 2.90GHz.
	}
\end{figure}

\subsection{\texttt{benchmarks-cpu-gpu}}

This benchmark measures the speedup achieved when the CPU/GPU strategy is used against pure CPU. To execute this test
Micropp should be compiled with OpenACC:

\begin{lstlisting}[language=Bash,backgroundcolor=\color{lightgray} ]
./benchmark-cpu-gpu
\end{lstlisting}

The output of this program is stored in \texttt{benchmark-cpu-gpu.dat}.

Fig.~\ref{fig:cpu-vs-gpu} shows the computing time of one Newton-Raphson iteration for the micro-scale problem
evaluating different mesh resolutions. The comparison is performed between one IBM Power 9 CPU core and an NVidia V100
Tesla GPU of the CTE-POWER cluster~\cite{cte-power} using \texttt{benchmark-cpu-gpu}~\cite{micropp-doc}. The
Newton-Raphson iteration includes the assembly of the Jacobian matrix, two times the assembly of the Residue vector (the
second assembly is done for checking the that the final norm is near zero after the solver) and the CGPD solver part.
This is the most fundamental procedure that it is done at the micro-scale in the linear or elastic cases when the
One-Way or the Full coupling schemes are activated. Moreover, in non-linear cases, this procedure is repeated several
times after the convergence is achieved. Fig.~\ref{fig:cpu-vs-gpu} demonstrates the speedup gained with the CPU/GPU
acceleration against using a pure CPU core. The speedup increases as the micro-scale mesh resolution increases, for
instance, a speedup of $\times$25 is gained for the mesh resolution of 100\tst elements. This increment in the speed up
with the mesh resolution is due to that the CGPD solver occupies a dominant part in the algorithm mainly for large mesh
resolutions. Being the solver the part which is better parallelized with the GPU the computations (principally the MVP
product) of the stage is accelerated and it is translated to the whole calculation.

\begin{figure}
	\centering
	\begin{tikzpicture}
		\pgfplotstableread[row sep=\\,col sep=&]{
			interval & CPU    & GPU  & Speedup  \\
			%10       & 53     & 176  & $\times$0.3  \\
			%20       & 496    & 349  & $\times$1.4  \\
			30       & 2.183   & 0.648  & $\times$3.3  \\
			40       & 6.221   & 1.146 & $\times$5.4  \\
			50       & 13.824  & 1.541 & $\times$9.0  \\
			60       & 24.751  & 1.464 & $\times$16.9 \\
			70       & 41.142  & 2.369 & $\times$17.4 \\
			80       & 73.224  & 4.204 & $\times$17.4 \\
			90       & 105.634 & 5.365 & $\times$19.7 \\
			100      & 182.470 & 7.315 & $\times$24.9 \\
		}\mydata
		\begin{axis}[
			grid=major,
			ybar=0.5cm,
			xlabel=Micro-Scale Mesh Resolution,
			x unit=\# Elements,
			ylabel=Computing Time,
			y unit=s,
			x=0.15cm,
			ymin=-1,
			xtick={30,40,50,60,70,80,90,100},
			xticklabels={30\tst,40\tst,50\tst,60\tst,70\tst,80\tst,90\tst,100\tst},
			x tick label style={anchor=east},
			scaled y ticks=false,
			legend style={anchor=north west},
			legend pos= north west
			]
			\addplot [fill=blue, ybar, point meta = explicit symbolic, nodes near coords]
				table[meta=Speedup,x=interval,y=CPU] {\mydata};
			\addplot [fill=green, ybar, point meta = explicit symbolic, nodes near coords]
				table[x=interval,y=GPU] {\mydata};
			\legend{CPU, GPU};
		\end{axis}
	\end{tikzpicture}
	\caption{\label{fig:cpu-vs-gpu}
		Computing Time of one Newton-Raphson iteration step in the micro-scale for different mesh resolutions
		(\texttt{benchmark-cpu-gpu}~\cite{micropp-doc}). Comparison between one CPU IBM Power 9 CPU core and an
		NVIDIA V100 GPU (computing node of CTE-POWER cluster).
	}
\end{figure}
\subsection{\texttt{benchmarks-elastic}}
\subsection{\texttt{benchmarks-plastic}}
\subsection{\texttt{benchmarks-damage}}
\subsection{\texttt{benchmarks-mic-1}}
\subsection{\texttt{benchmarks-mic-2}}
\subsection{\texttt{benchmarks-mic-3}}
