
\section{Benchmarks}


\subsection{\texttt{benchmarks-sol-ass}}

This benchmark is intended to measure the computing times of the assembly of the Residue vector ($\times2$) and the
Jacobian matrix and the solver (CGPD).

Basically the benchmark executes a tipical Newton-Raphson iteration:

\begin{lstlisting}
	double norm = assembly_rhs_acc(u, nullptr, b);

#ifdef _OPENACC
	assembly_mat_acc(&A, u, nullptr);
#else
	assembly_mat(&A, u, nullptr);
#endif

#ifdef _OPENACC
	int cg_its = ell_solve_cgpd_acc(&A, b, du, &cg_err);
#else
	int cg_its = ell_solve_cgpd(&A, b, du, &cg_err);
#endif

	for (int i = 0; i < nn * dim; ++i)
		u[i] += du[i];

	norm = assembly_rhs_acc(u, nullptr, b);
\end{lstlisting}

\begin{figure}[!htbp]
	\centering
	\begin{tikzpicture}[]
		\pgfplotsset{every tick label/.append style={font=\small}}
		\pgfplotstableread{data/benchmark-sol-ass-macintosh.dat}{\times}
		\begin{axis}[
%			grid=major,
%			y unit=s,
%			legend pos=north west,
%			legend cell align={left},
%			ylabel=Computing Time,
%			xlabel=Micro-resolution,
%			x unit=\# Elements,
%			ymin = 0,
%			ymax = 600,
%			xtick = {0,20,40,60,80,100},
%			xticklabels = {0,20\tst,40\tst,60\tst,80\tst,100\tst},
%			ytick = {0,200,400,600,800},
			]
			\addplot [color=blue,mark=*,line width = 0.5mm] table [x index={0}, y index={1}] {\times};
			\addplot [color=green,mark=*,line width = 0.5mm] table [x index={0}, y index={2}] {\times};
%				%{\times} [yshift=9pt] 
%				%node[pos=0.0,yshift=10pt] {36\%}
%				%node[pos=0.105,yshift=11pt] {68\%}
%				%node[pos=1.0] {79\%};
%			%\addplot [color=blue ,mark=*,line width = 0.5mm]  table [x={n1}, y expr=\thisrowno{2}*1.0e-6]
%				%{\times} [yshift=9pt] 
%				%node[pos=0.0] {64\%}
%				%node[pos=0.253,yshift=5pt] {32\%}
%				%node[pos=1.0] {21\%};
%			%\legend{Jacobian \& Residue Assembly}
		\end{axis}
	\end{tikzpicture}
%	\caption{\label{fig:ass_vs_sol}
%		Computing time used for the assembly of the Jacobian Matrix and the Residue vector and the solver
%		algorithm of the Micropp code to perform the micro-scale FE calculation.
%	}
\end{figure}

\subsection{\texttt{benchmarks-cpu-gpu}}
\subsection{\texttt{benchmarks-elastic}}
\subsection{\texttt{benchmarks-plastic}}
\subsection{\texttt{benchmarks-damage}}
\subsection{\texttt{benchmarks-mic-1}}
\subsection{\texttt{benchmarks-mic-2}}
\subsection{\texttt{benchmarks-mic-3}}
